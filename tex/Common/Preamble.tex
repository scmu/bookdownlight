%%%%%%%%%%%%%%%%%%%%%%%%%%%%%%%%%%%%
% Imported packages

\usepackage{xcolor}
\usepackage{xspace}
\usepackage{amsmath}
\usepackage{amsthm}
%\usepackage{amssymb}
\usepackage{Common/doubleequals}
\usepackage{comment}
\usepackage{url}
\usepackage{natbib}
\usepackage{paralist}
\usepackage[framemethod=TikZ]{mdframed}
\usepackage{multicol}
\usepackage{tikz}  % for chapterstyle
\usepackage{mathptmx}
%\usepackage{helvet}
\usepackage{scalerel}

\usepackage[T1]{fontenc}
\usepackage[sfdefault]{biolinum}
\renewcommand{\familydefault}{\sfdefault}

%%%%%%%%%%%%%%%%%%%%%%%%%%%%%%%%%%%%
% for the greek and mayan numbers

\usepackage[main=english, polutonikogreek]{babel}
\usepackage{textgreek}
%\usepackage{teubner}
%\usepackage{subfigure}
%\usepackage{graphicx,MnSymbol}

\DeclareFontFamily{U}{mathb}{\hyphenchar\font45}
\DeclareFontShape{U}{mathb}{m}{n}{
      <5> <6> <7> <8> <9> <10> gen * mathb
      <10.95> mathb10 <12> <14.4> <17.28> <20.74> <24.88> mathb12
      }{}
\DeclareSymbolFont{mathb}{U}{mathb}{m}{n}

\newcommand\mathbfont{\usefont{U}{mathb}{m}{n}}

\newcount\mayacnta%
\newcount\mayacntc%
%
\def\mayaexpansion{%
    \mayacntc=\mayacnta\mathbfont
    \ifnum\mayacntc=0 0\else
    \rotatebox[origin=c]{-90}{%
    \loop\ifnum\mayacntc>5\advance\mayacntc by -5\repeat
    \the\mayacntc\mayacntc=\mayacnta
    \loop\ifnum\mayacntc>5\advance\mayacntc by -5 5\repeat}%
    \fi}%
\def\mayadigit#1{%
    \mayacnta=#1\hbox{\mathbfont\mayaexpansion}}%


\DeclareMathAlphabet{\mymathbb}{U}{BOONDOX-ds}{m}{n} % for tt and ff in Derivation.lhs

\setsecnumdepth{subsection}  %number subsections

%%%%%%%%%%%%%%%%%%%%%%%%%%%%%%%%%%%%
% Exercises
\usepackage[answerdelayed]{exercise}
%\usepackage[noanswer]{exercise} % to suppress answers
\numberwithin{Answer}{chapter}
\numberwithin{Exercise}{chapter}
\renewcommand{\ExerciseListName}{習題}
\setlength{\Exetopsep}{15pt}
\setlength{\Exesep}{5pt}

%%%%%%%%%%%%%%%%%%%%%%%%%%%%%%%%%%%%
% Set Chinese Font

%\usepackage{fontspec}
\usepackage{xeCJK}
\renewcommand{\familydefault}{\sfdefault} % use sans serif font
\setCJKmainfont[BoldFont=STHeiti,ItalicFont=STKaiti]{STSong}
\setCJKsansfont[BoldFont=Heiti TC Medium,ItalicFont=STFangsong]{Heiti TC Light}%{STXihei}
%\setCJKsansfont[BoldFont=STHeiti,ItalicFont=STKaiti]{Heiti TC Light}%{STXihei}
\setCJKmonofont{STFangsong}
\renewcommand{\baselinestretch}{1.15}
%\setlength{\jot}{-10pt}

%\setromanfont[BoldFont={BiauKai}]{Apple LiSung Light}

%%%%%%%%%%%%%%%%%%%%%%%%%%%%%%%%%%%%
% Preamble Extracted from lhs2Tex.
%   Bruno's trick for separate type-setting.

%%include lhs2TeX.fmt
%% ODER: format ==         = "\mathrel{==}"
%% ODER: format /=         = "\neq "
%
%
\makeatletter
\@ifundefined{lhs2tex.lhs2tex.sty.read}%
  {\@namedef{lhs2tex.lhs2tex.sty.read}{}%
   \newcommand\SkipToFmtEnd{}%
   \newcommand\EndFmtInput{}%
   \long\def\SkipToFmtEnd#1\EndFmtInput{}%
  }\SkipToFmtEnd

\newcommand\ReadOnlyOnce[1]{\@ifundefined{#1}{\@namedef{#1}{}}\SkipToFmtEnd}
\usepackage{amstext}
\usepackage{amssymb}
\usepackage{stmaryrd}
\DeclareFontFamily{OT1}{cmtex}{}
\DeclareFontShape{OT1}{cmtex}{m}{n}
  {<5><6><7><8>cmtex8
   <9>cmtex9
   <10><10.95><12><14.4><17.28><20.74><24.88>cmtex10}{}
\DeclareFontShape{OT1}{cmtex}{m}{it}
  {<-> ssub * cmtt/m/it}{}
\newcommand{\texfamily}{\fontfamily{cmtex}\selectfont}
\DeclareFontShape{OT1}{cmtt}{bx}{n}
  {<5><6><7><8>cmtt8
   <9>cmbtt9
   <10><10.95><12><14.4><17.28><20.74><24.88>cmbtt10}{}
\DeclareFontShape{OT1}{cmtex}{bx}{n}
  {<-> ssub * cmtt/bx/n}{}
\newcommand{\tex}[1]{\text{\texfamily#1}}	% NEU

\newcommand{\Sp}{\hskip.33334em\relax}


\newcommand{\Conid}[1]{\mathit{#1}}
\newcommand{\Varid}[1]{\mathit{#1}}
\newcommand{\anonymous}{\kern0.06em \vbox{\hrule\@width.5em}}
\newcommand{\plus}{\mathbin{+\!\!\!+}}
\newcommand{\bind}{\mathbin{>\!\!\!>\mkern-6.7mu=}}
\newcommand{\rbind}{\mathbin{=\mkern-6.7mu<\!\!\!<}}% suggested by Neil Mitchell
\newcommand{\sequ}{\mathbin{>\!\!\!>}}
\renewcommand{\leq}{\leqslant}
\renewcommand{\geq}{\geqslant}
\usepackage{polytable}

%mathindent has to be defined
\@ifundefined{mathindent}%
  {\newdimen\mathindent\mathindent\leftmargini}%
  {}%

\def\resethooks{%
  \global\let\SaveRestoreHook\empty
  \global\let\ColumnHook\empty}
\newcommand*{\savecolumns}[1][default]%
  {\g@addto@macro\SaveRestoreHook{\savecolumns[#1]}}
\newcommand*{\restorecolumns}[1][default]%
  {\g@addto@macro\SaveRestoreHook{\restorecolumns[#1]}}
\newcommand*{\aligncolumn}[2]%
  {\g@addto@macro\ColumnHook{\column{#1}{#2}}}

\resethooks

\newcommand{\onelinecommentchars}{\quad-{}- }
\newcommand{\commentbeginchars}{\enskip\{-}
\newcommand{\commentendchars}{-\}\enskip}

\newcommand{\visiblecomments}{%
  \let\onelinecomment=\onelinecommentchars
  \let\commentbegin=\commentbeginchars
  \let\commentend=\commentendchars}

\newcommand{\invisiblecomments}{%
  \let\onelinecomment=\empty
  \let\commentbegin=\empty
  \let\commentend=\empty}

\visiblecomments

\newlength{\blanklineskip}
\setlength{\blanklineskip}{0.66084ex}

\newcommand{\hsindent}[1]{\quad}% default is fixed indentation
\let\hspre\empty
\let\hspost\empty
\newcommand{\NB}{\textbf{NB}}
\newcommand{\Todo}[1]{$\langle$\textbf{To do:}~#1$\rangle$}

\EndFmtInput
\makeatother
%
%
%
%
%
%
% This package provides two environments suitable to take the place
% of hscode, called "plainhscode" and "arrayhscode". 
%
% The plain environment surrounds each code block by vertical space,
% and it uses \abovedisplayskip and \belowdisplayskip to get spacing
% similar to formulas. Note that if these dimensions are changed,
% the spacing around displayed math formulas changes as well.
% All code is indented using \leftskip.
%
% Changed 19.08.2004 to reflect changes in colorcode. Should work with
% CodeGroup.sty.
%
\ReadOnlyOnce{polycode.fmt}%
\makeatletter

\newcommand{\hsnewpar}[1]%
  {{\parskip=0pt\parindent=0pt\par\vskip #1\noindent}}

% can be used, for instance, to redefine the code size, by setting the
% command to \small or something alike
\newcommand{\hscodestyle}{}

% The command \sethscode can be used to switch the code formatting
% behaviour by mapping the hscode environment in the subst directive
% to a new LaTeX environment.

\newcommand{\sethscode}[1]%
  {\expandafter\let\expandafter\hscode\csname #1\endcsname
   \expandafter\let\expandafter\endhscode\csname end#1\endcsname}

% "compatibility" mode restores the non-polycode.fmt layout.

\newenvironment{compathscode}%
  {\par\noindent
   \advance\leftskip\mathindent
   \hscodestyle
   \let\\=\@normalcr
   \(\pboxed}%
  {\endpboxed\)%
   \par\noindent
   \ignorespacesafterend}

\newcommand{\compaths}{\sethscode{compathscode}}

% "plain" mode is the proposed default.

\newenvironment{plainhscode}%
  {\hsnewpar\abovedisplayskip
   \advance\leftskip\mathindent
   \hscodestyle
   \let\\=\@normalcr
   \(\pboxed}%
  {\endpboxed\)%
   \hsnewpar\belowdisplayskip
   \ignorespacesafterend}

% Here, we make plainhscode the default environment.

\newcommand{\plainhs}{\sethscode{plainhscode}}
\plainhs

% The arrayhscode is like plain, but makes use of polytable's
% parray environment which disallows page breaks in code blocks.

\newenvironment{arrayhscode}%
  {\hsnewpar\abovedisplayskip
   \advance\leftskip\mathindent
   \hscodestyle
   \let\\=\@normalcr
   \(\parray}%
  {\endparray\)%
   \hsnewpar\belowdisplayskip
   \ignorespacesafterend}

\newcommand{\arrayhs}{\sethscode{arrayhscode}}

% The mathhscode environment also makes use of polytable's parray 
% environment. It is supposed to be used only inside math mode 
% (I used it to typeset the type rules in my thesis).

\newenvironment{mathhscode}%
  {\parray}{\endparray}

\newcommand{\mathhs}{\sethscode{mathhscode}}

% texths is similar to mathhs, but works in text mode.

\newenvironment{texthscode}%
  {\(\parray}{\endparray\)}

\newcommand{\texths}{\sethscode{texthscode}}

% The framed environment places code in a framed box.

\def\codeframewidth{\arrayrulewidth}
\RequirePackage{calc}

\newenvironment{framedhscode}%
  {\parskip=\abovedisplayskip\par\noindent
   \hscodestyle
   \arrayrulewidth=\codeframewidth
   \tabular{@{}|p{\linewidth-2\arraycolsep-2\arrayrulewidth-2pt}|@{}}%
   \hline\framedhslinecorrect\\{-1.5ex}%
   \let\endoflinesave=\\
   \let\\=\@normalcr
   \(\pboxed}%
  {\endpboxed\)%
   \framedhslinecorrect\endoflinesave{.5ex}\hline
   \endtabular
   \parskip=\belowdisplayskip\par\noindent
   \ignorespacesafterend}

\newcommand{\framedhslinecorrect}[2]%
  {#1[#2]}

\newcommand{\framedhs}{\sethscode{framedhscode}}

% The inlinehscode environment is an experimental environment
% that can be used to typeset displayed code inline.

\newenvironment{inlinehscode}%
  {\(\def\column##1##2{}%
   \let\>\undefined\let\<\undefined\let\\\undefined
   \newcommand\>[1][]{}\newcommand\<[1][]{}\newcommand\\[1][]{}%
   \def\fromto##1##2##3{##3}%
   \def\nextline{}}{\) }%

\newcommand{\inlinehs}{\sethscode{inlinehscode}}

% The joincode environment is a separate environment that
% can be used to surround and thereby connect multiple code
% blocks.

\newenvironment{joincode}%
  {\let\orighscode=\hscode
   \let\origendhscode=\endhscode
   \def\endhscode{\def\hscode{\endgroup\def\@currenvir{hscode}\\}\begingroup}
   %\let\SaveRestoreHook=\empty
   %\let\ColumnHook=\empty
   %\let\resethooks=\empty
   \orighscode\def\hscode{\endgroup\def\@currenvir{hscode}}}%
  {\origendhscode
   \global\let\hscode=\orighscode
   \global\let\endhscode=\origendhscode}%

\makeatother
\EndFmtInput
%


%%%%%%%%%%%%%%%%%%%%%%%%%%%%%%%%%%%%
% Theorems & Lemma numbering

\newtheorem{theorem}{\color{nicered}{定理}}[chapter]
\newtheorem{lemma}[theorem]{\color{nicered}{引理}}
\newtheorem{example}[theorem]{\color{nicered}{例}}
\newtheorem{definition}[theorem]{\color{nicered}{定義}}

%%%%%%%%%%%%%%%%%%%%%%%%%%%%%%%%%%%%
% Lhs2Tex Comment Style
%  In Formatting.fmt

% \def\commentbegin{\quad\{\ }
% \def\commentend{\}}

%%%%%%%%%%%%%%%%%%%%%%%%%%%%%%%%%%%%
% Box and Chapter Style

\definecolor{verylightgray}{rgb}{0.9, 0.9, 0.9}
\newmdenv[roundcorner=5pt, innerbottommargin=10pt,
linecolor=verylightgray, backgroundcolor=verylightgray]{infoshade}

\newenvironment{infobox}[1]
{\begin{figure}[t]
\begin{infoshade}[frametitle=#1]
\small
}
{
\end{infoshade}
\end{figure}
}

\definecolor{veryverylightgray}{rgb}{0.95, 0.95, 0.95}
\newmdenv[roundcorner=2pt, innerbottommargin=7pt,
linecolor=veryverylightgray, backgroundcolor=veryverylightgray]{exbox}

\newenvironment{exlist}
{
\begin{exbox}
\begin{minipage}{\textwidth}
\color{black}
\begin{ExerciseList}
}
{
\end{ExerciseList}
\end{minipage}
\end{exbox}
}

\newenvironment{answer}
{
\begin{proof}[\color{nicered}{\bf 答}]
}
{
\renewcommand{\qedsymbol}{$\color{nicered}{\Box}$}
%\renewcommand{\qedsymbol}{$\blacksquare$}
\end{proof}
}


\newcommand\todo[1]{\textcolor{blue!70!gray}{[todo: #1]}}

% https://hstuart.dk/2007/05/21/styling-the-chapter/
% https://texblog.org/2012/07/03/fancy-latex-chapter-styles/

\makechapterstyle{box}{
\renewcommand*{\printchaptername}{} \renewcommand*{\chapnumfont}{\normalfont\sffamily\huge\bfseries} \renewcommand*{\printchapternum}{ \flushright \begin{tikzpicture} \draw[fill,color=black] (0,0) rectangle (2cm,2cm); \draw[color=white] (1cm,1cm) node { \chapnumfont\thechapter }; \end{tikzpicture} } \renewcommand*{\chaptitlefont}{\normalfont\sffamily\Huge\bfseries} \renewcommand*{\printchaptertitle}[1]{\flushright\chaptitlefont##1} }

\makechapterstyle{box}{ \renewcommand*{\printchaptername}{} \renewcommand*{\chapnumfont}{\normalfont\sffamily\huge\bfseries} \renewcommand*{\printchapternum}{ \flushright \begin{tikzpicture} \draw[fill,color=black] (0,0) rectangle (2cm,2cm); \draw[color=white] (1cm,1cm) node { \chapnumfont\thechapter }; \end{tikzpicture} } \renewcommand*{\chaptitlefont}{\normalfont\sffamily\Huge\bfseries} \renewcommand*{\printchaptertitle}[1]{\flushright\chaptitlefont##1} }

%http://ftp.yzu.edu.tw/CTAN/info/latex-samples/MemoirChapStyles/MemoirChapStyles.pdf

%\usepackage{color,calc,graphicx,soul}
\usepackage{color,calc,graphicx,soul}
%\usepackage[widespace]{fourier}
\definecolor{nicered}{rgb}{.647,.129,.149}
\makeatletter
\newlength\dlf@normtxtw
\setlength\dlf@normtxtw{\textwidth}
\def\myhelvetfont{\def\sfdefault{mdput}}
\newsavebox{\feline@chapter}
\newcommand\feline@chapter@marker[1][4cm]{%
\sbox\feline@chapter{%
\resizebox{!}{#1}{\fboxsep=1pt%
\colorbox{nicered}{\color{white}\bfseries\sffamily\thechapter}%
}}%
\rotatebox{90}{%
\resizebox{%
\heightof{\usebox{\feline@chapter}}+\depthof{\usebox{\feline@chapter}}}%
{!}{\scshape\so\@chapapp}}\quad%
\raisebox{\depthof{\usebox{\feline@chapter}}}{\usebox{\feline@chapter}}%
}
\newcommand\feline@chm[1][4cm]{%
\sbox\feline@chapter{\feline@chapter@marker[#1]}%
\makebox[0pt][l]{% aka \rlap
\makebox[1cm][r]{\usebox\feline@chapter}%
}}
\makechapterstyle{daleif1}{
\renewcommand\chapnamefont{\normalfont\Large\scshape\raggedleft\so}
\renewcommand\chaptitlefont{\normalfont\huge\bfseries\scshape\color{nicered}}
\renewcommand\chapternamenum{}
\renewcommand\printchaptername{}
\renewcommand\printchapternum{\null\hfill\feline@chm[2.5cm]\par}
\renewcommand\afterchapternum{\par\vskip\midchapskip}
\renewcommand\printchaptertitle[1]{\chaptitlefont\raggedleft ##1\par}
}
\makeatother
%\chapterstyle{daleif1}

% \usepackage{fourier} % or what ever
%\usepackage[scaled=.92]{helvet}%. Sans serif - Helvetica
\usepackage{color,calc}
\newsavebox{\ChpNumBox}
\definecolor{ChapBlue}{rgb}{0.00,0.65,0.65}
\makeatletter
\newcommand*{\thickhrulefill}{%
\leavevmode\leaders\hrule height 1\p@ \hfill \kern \z@}
\newcommand*\BuildChpNum[2]{%
\begin{tabular}[t]{@{}c@{}}
\makebox[0pt][c]{#1\strut} \\[.5ex]
\colorbox{ChapBlue}{%
\rule[-10em]{0pt}{0pt}%
\rule{1ex}{0pt}\color{black}#2\strut
\rule{1ex}{0pt}}%
\end{tabular}}
\makechapterstyle{BlueBox}{%
\renewcommand{\chapnamefont}{\large\scshape}
\renewcommand{\chapnumfont}{\Huge\bfseries}
\renewcommand{\chaptitlefont}{\raggedright\Huge\bfseries}
\setlength{\beforechapskip}{20pt}
\setlength{\midchapskip}{26pt}
\setlength{\afterchapskip}{40pt}
\renewcommand{\printchaptername}{}
\renewcommand{\chapternamenum}{}
\renewcommand{\printchapternum}{%
\sbox{\ChpNumBox}{%

\BuildChpNum{\chapnamefont\@chapapp}%
{\chapnumfont\thechapter}}}
\renewcommand{\printchapternonum}{%
\sbox{\ChpNumBox}{%
\BuildChpNum{\chapnamefont\vphantom{\@chapapp}}%
{\chapnumfont\hphantom{\thechapter}}}}
\renewcommand{\afterchapternum}{}
\renewcommand{\printchaptertitle}[1]{%
\usebox{\ChpNumBox}\hfill
\parbox[t]{\hsize-\wd\ChpNumBox-1em}{%
\vspace{\midchapskip}%
\thickhrulefill\par
\chaptitlefont ##1\par}}%
}
%\chapterstyle{BlueBox}

\chapterstyle{daleif1}
\setsecheadstyle{\color{nicered}\Large\bfseries\raggedright}
\setparaheadstyle{\color{nicered}\bfseries\raggedright}
%\setsechook{\color{nicered}}

%%%%%%%%%%%%%%%%%%%%%%%%%%%%%%%%%%%%
% Math Color

  % blues
\definecolor{prussianblue}{rgb}{0.0, 0.19, 0.33}
\definecolor{lapislazuli}{rgb}{0.15, 0.38, 0.61}
\definecolor{tealblue}{rgb}{0.21, 0.46, 0.53}

  % oranges
\definecolor{terracotta}{rgb}{0.89, 0.45, 0.36}
\definecolor{burntorange}{rgb}{0.8, 0.33, 0.0}

\everymath{\color{tealblue}}
%to avoid fake math mathcoloring
%   https://tex.stackexchange.com/questions/100263/everymath-and-author-color
% \makeatletter  \def\m@th{\mathsurround\z@\color{black}} \makeatother

%\def\arraystretch{0}
%\XeTeXlinebreaklocale "zh"
%\XeTeXlinebreakskip = 0pt plus 1pt

%%%%%%%%%%%%%%%%%%%%%%%%%%%%%%%%%%%%
% Draft Watermark

\usepackage{draftwatermark}
\SetWatermarkText{DRAFT}
\SetWatermarkScale{1}
